\documentclass[convert={density=300,size=1200x800,outext=.png}]{standalone}

\usepackage{tikz}
\usepackage{ifthen}
\begin{document}
\pagestyle{empty}

% Custom definitions -- START
\def\layersep{0}
\def\inactiveOpacity{0.3}
\def\layernodes{5}
\def\activenode{3}
\def\displaymode{1} % 1 for forward pass, 2 for backward pass

\tikzstyle{every pin edge}=[<-,shorten <=1pt]
\tikzstyle{neuron}=[circle,opacity=0.3,fill=black!25,minimum size=17pt,inner sep=0pt]
\tikzstyle{input neuron}=[neuron, fill=green!35]
\tikzstyle{output neuron}=[neuron, fill=cyan!50]
\tikzstyle{inactive}=[opacity=\inactiveOpacity]
\tikzstyle{active}=[opacity=1]
\tikzstyle{hide}=[opacity=0]
% Custom definitions -- END

% Comparison of the network connectivity in general neural networks and convolutional neural networks
\begin{tikzpicture}[shorten >=1pt,->,opacity=\inactiveOpacity,draw=black!50]
    % NODES -- START
    
    % Neural networks -- START
    \foreach \i in {1, ...,\layernodes}
        \ifthenelse{\displaymode=2 \OR \i=\activenode}{\def\status{active}}{\def\status{}}
        \node[input neuron,\status] (NN-I-\i) at (0,-1.2 - \i) {\small $I_\i$};

    \foreach \i in {1, ...,\layernodes}
        \ifthenelse{\displaymode=1 \OR \i=\activenode}{\def\status{active}}{\def\status{}}
        \node[output neuron,\status] (NN-O-\i) at (2,-1.2 - \i) {\small $O_\i$};
   
    % connections
    \foreach \source in {1,...,\layernodes}
        \foreach \dest in {1,...,\layernodes}
            \ifthenelse{\displaymode=2}{\def\target{\dest}}{\def\target{\source}}
            \ifthenelse{\target=\activenode}{\def\status{active}}{\def\status{}}
            \path (NN-I-\source) edge[\status] (NN-O-\dest);
    % Neural networks -- END

    % Convolutional Neural Networks -- START
    \foreach \i in {1, ...,\layernodes}
        \ifthenelse{\i=\activenode}{\def\status{active}}{\def\status{}}
        \node[input neuron,\status] (CNN-I-\i) at (5,-1.2 - \i) {\small $I_\i$};

    \foreach \i in {1, ...,\layernodes}
        \ifthenelse{\i=2 \OR \i=3 \OR \i=4}{\def\status{active}}{\def\status{}}
        \node[output neuron,\status] (CNN-O-\i) at (7,-1.2 - \i) {\small $O_\i$};
  
    \path (CNN-I-1) edge (CNN-O-1);
    \path (CNN-I-1) edge (CNN-O-2);
    
    \path (CNN-I-2) edge (CNN-O-1);
    \path (CNN-I-2) edge (CNN-O-2);
    \path (CNN-I-2) edge (CNN-O-3);

    \path (CNN-I-3) edge[active] (CNN-O-2);
    \path (CNN-I-3) edge[active] (CNN-O-3);
    \path (CNN-I-3) edge[active] (CNN-O-4);
    
    \path (CNN-I-4) edge (CNN-O-3);
    \path (CNN-I-4) edge (CNN-O-4);
    \path (CNN-I-4) edge (CNN-O-5);

    \path (CNN-I-4) edge (CNN-O-3);
    \path (CNN-I-4) edge (CNN-O-4);
    \path (CNN-I-4) edge (CNN-O-5);

    \path (CNN-I-5) edge (CNN-O-4);
    \path (CNN-I-5) edge (CNN-O-5);

   % \foreach \source in {1,...,\layernodes}
    %    \foreach \dest in {-1,...,1}
     %       \pgfmathsetmacro \test {\source+\dest}
      %      \ifthenelse{\test > 0 \AND \test < 6}{\path (CNN-I-\source) edge (CNN-O-\test)}{}
            %\ifnum\value{@test} <= \value{layernodes}
      %          \path (CNN-I-\source) edge (CNN-O-\test);
            %\fi
    % Convolutional Neural networks -- END
\end{tikzpicture}

\end{document}

